\documentclass{article}

\usepackage{amsmath,amssymb}
\usepackage{graphicx}
\usepackage[margin=1in]{geometry}
\usepackage{dsfont}

% ===== This makes my \affil cmnd work.
\usepackage[affil-it]{authblk}


% ===== This makes my environments work switching llncs to article.
\newtheorem{theorem}{Theorem}[section]
\newtheorem{lemma}[theorem]{Lemma}
\newtheorem{proposition}[theorem]{Proposition}
\newtheorem{corollary}[theorem]{Corollary}

\newenvironment{proof}[1][Proof]{\begin{trivlist}
\item[\hskip \labelsep {\bfseries #1}]}{\end{trivlist}}
\newenvironment{definition}[1][Definition]{\begin{trivlist}
\item[\hskip \labelsep {\bfseries #1}]}{\end{trivlist}}
\newenvironment{example}[1][Example]{\begin{trivlist}
\item[\hskip \labelsep {\bfseries #1}]}{\end{trivlist}}
\newenvironment{remark}[1][Remark]{\begin{trivlist}
\item[\hskip \labelsep {\bfseries #1}]}{\end{trivlist}}

\newcommand{\qed}{\nobreak \ifvmode \relax \else
      \ifdim\lastskip<1.5em \hskip-\lastskip
      \hskip1.5em plus0em minus0.5em \fi \nobreak
      \vrule height0.75em width0.5em depth0.25em\fi}

% ===== For \algorithm. Is this a decent idea?
\usepackage[lined,boxed,ruled,vlined]{algorithm2e}

% ===== For \mathscr
\usepackage{mathrsfs}
\DeclareSymbolFontAlphabet{\mathrsfs}{rsfs}
\usepackage[mathscr]{eucal}


% ===== For \boldsymbol
\usepackage{amsbsy}

% ===== For \bm (bold math)
\usepackage{bm}

\usepackage{fixltx2e}
\MakeRobust{\overrightarrow}

% ==== Misha and Ning's Notation file =====
%% ----------------------------------------------------------------------
%% Definitions, Macros, Etc.
%% ----------------------------------------------------------------------

%% Hyper-linked References
\newcommand{\Sec}[1]{\hyperref[sec:#1]{\S\ref*{sec:#1}}} %section
\newcommand{\Eqn}[1]{\hyperref[eq:#1]{(\ref*{eq:#1})}} %equation
\newcommand{\Fig}[1]{\hyperref[fig:#1]{Figure~\ref*{fig:#1}}} %figure
\newcommand{\Tab}[1]{\hyperref[tab:#1]{Table~\ref*{tab:#1}}} %table
\newcommand{\Thm}[1]{\hyperref[thm:#1]{Theorem~\ref*{thm:#1}}} %theorem
\newcommand{\Lem}[1]{\hyperref[lem:#1]{Lemma~\ref*{lem:#1}}} %lemma
\newcommand{\Prop}[1]{\hyperref[prop:#1]{Property~\ref*{prop:#1}}} %property
\newcommand{\Cor}[1]{\hyperref[cor:#1]{Corollary~\ref*{cor:#1}}} %corollary
\newcommand{\Def}[1]{\hyperref[def:#1]{Definition~\ref*{def:#1}}} %definition
\newcommand{\Alg}[1]{\hyperref[alg:#1]{Algorithm~\ref*{alg:#1}}} %algorithm
\newcommand{\Ex}[1]{\hyperref[ex:#1]{Example~\ref*{ex:#1}}} %example

% Theorem-like constructs
%\newtheorem{example}[theorem]{Example}

% Blackboard fonts 
\newcommand{\Real}{\mathbb{R}}
\newcommand{\Cplx}{\mathbb{C}}
%% Transposes
\newcommand{\Tra}{^{\rm T}} % Transpose
\newcommand{\Cct}{^\dagger} % Complex conjugate transpose

%% Permutation index
\newcommand{\bfpp}{{\bf p}_n}

%% Matrix & Tensor Operations
\newcommand{\Circ}[1]{{\rm circ}\left( #1 \right)}
\newcommand{\Fold}[1]{{\rm fold}\left( #1 \right)}
\newcommand{\Unfold}[1]{{\rm unfold}\left( #1 \right)}
\newcommand{\Twist}[1]{{\rm twist}(\M{#1})}
\newcommand{\Squeeze}[1]{{\rm squeeze}(#1)}
\newcommand{\squeeze}{{\rm squeeze}}
\newcommand{\Mout}{\diamondsuit}
\newcommand{\circu}{ {\rm circ}}
\newcommand{\bcirc}{ {\rm circ}}
\newcommand{\vvec}{ {\rm vec}}

\newcommand{\mc}[1]{\mathcal{#1}}
\newcommand{\mb}[1]{\mathbb{#1}}
\newcommand{\mcr}[1]{\mathrsfs{#1}}

%% Element of complicated object that is surrounded by parens
\newcommand{\PE}[2]{\left( #1 \right)_{#2}}

%% Vector notation
\newcommand{\V}[1]{{\bm{\mathbf{\MakeLowercase{#1}}}}} % vector
\newcommand{\VE}[2]{\MakeLowercase{#1}_{#2}} % vector element

%% Matrix notation
\newcommand{\M}[1]{{\bm{\mathbf{\MakeUppercase{#1}}}}} % matrix
\newcommand{\Mhat}[1]{{\bm{\hat \mathbf{\MakeUppercase{#1}}}}} % matrix
\newcommand{\Mbar}[1]{{\bm{\bar \mathbf{\MakeUppercase{#1}}}}} % matrix
\newcommand{\ME}[2]{\MakeLowercase{#1}_{#2}} % matrix element
\newcommand{\MC}[2]{\V{#1}_{#2}}

%% Tensor notation
\newcommand{\T}[1]{\boldsymbol{\mathscr{\MakeUppercase{#1}}}} %tensor
\newcommand{\TLS}[2]{\M{#1}_{[#2]}} % lateral slice
\newcommand{\TFS}[2]{\M{#1}_{#2}} % frontal slice
\newcommand{\TT}[2]{\V{#1}_{#2}} % tube
\newcommand{\TE}[2]{\MakeLowercase{#1}_{#2}} % tensor element


%% Shortcuts
\newcommand{\TA}{\T{A}}
\newcommand{\TB}{\T{B}}
\newcommand{\TS}{\T{S}}
\newcommand{\TC}{\T{C}}
\newcommand{\TU}{\T{U}}
\newcommand{\TV}{\T{V}}
\newcommand{\TG}{\T{G}}

\newcommand{\Vu}{\V{u}}
\newcommand{\Vv}{\V{v}}
\newcommand{\Vq}{\V{q}}
\newcommand{\Vr}{\V{r}}
\newcommand{\Vp}{\V{p}}
\newcommand{\Vd}{\V{d}}
\newcommand{\Vzz}{\V{z}}
\newcommand{\Vb}{\V{b}}
\newcommand{\Vg}{\V{g}}
\newcommand{\Vh}{\V{h}}
\newcommand{\MH}{\M{H}}
\newcommand{\MG}{\M{G}}
\newcommand{\MA}{\M{A}}
\newcommand{\MX}{\M{X}}
\newcommand{\MZ}{\M{Z}}
\newcommand{\MW}{\M{W}}
%\newcommand{\TD}{\T{D}}

\newcommand{\SaS}{{\mathcal S}}

\newcommand{\MGC}{\tilde{\MG}}

\newcommand{\Matlab}{{\sc Matlab}\xspace}
\newcommand{\matlab}{{\sc Matlab}\xspace}
\newcommand{\qtext}[1]{\quad\text{#1}\quad}

\newcommand{\matvec}{{\tt Vec}}
\newcommand{\fld}{{\tt Fold}}

\def \bK{\mathbf{K}}
\def \bF{\mathbf{F}}
\def \bD{\mathbf{D}}
\def \bB{\mathbf{B}}
\def \bA{\mathbf{A}}
\newcommand{\bDelta}{\boldsymbol{\Delta}}

%\newcommand{\bea}{\left[ \begin{array}}
%\newcommand{\eea}{ \end{array} \right]} 

\newcommand{\bftheta}{ {\boldsymbol \theta}}
\newcommand{\bfrho}{ {\boldsymbol \rho}}
\newcommand{\bfeta}{ {\boldsymbol \eta}}
\newcommand{\fft}{ \mbox{\tt fft} }
\newcommand{\ifft}{ \mbox{\tt ifft} }
\newcommand{\blkd}{\mbox{\tt blkdiag}}
\newcommand{\rshpT}{\mbox{\tt reshapeT}}
\newcommand{\F}[1]{\mathcal{F}\{#1\}}
\newcommand{\Fi}[1]{\mathcal{F}^{-1}\{#1\}}



%%%% Dr. K's colored comments. 
\usepackage{color} 
\definecolor{blue}{rgb}{0,0,1}
\definecolor{red}{rgb}{1,0,0}
\definecolor{purple}{rgb}{1,0,1}
\newcommand\MEK[1]{\textcolor{red}{MEK: #1}}
\newcommand\EMK[1]{\textcolor{purple}{EMK: #1}}
\newcommand\SA[1]{\textcolor{blue}{SA: #1}}

\begin{document}


\title{Basis Expansion Monte Carlo}

\author{Eric Kernfeld
  \thanks{Electronic address: \texttt{ekernf01@u.washington.edu}; Corresponding author}}
\affil{University of Washington, Seattle, WA, USA}
\maketitle

\begin{abstract}
We introduce Basis Expansion Monte Carlo, which studies a Gibbs or Metropolis-Hastings sampler to infer the underlying transition kernel. To make inference about the steady state, we compute the steady-state of the approximate kernel. Results show ...
\end{abstract}


\section{Introduction}
In many statistical models, it is impossible to find a closed form for the distribution of interest (we will call this $\pi$). One work-around, originating in computational physics, relies on the fact that for points $x_1$ and $x_2$ in the parameter space, $\pi(x_1)/\pi(x_2)$ may still be calculable, though $\pi(x_1)$ and $\pi(x_2)$ are not. This fact is exploited to produce a Markov chain whose steady-state distribution is guaranteed to be $\pi$. 

More and references about history, background, and/or tutorials on monte carlo methods



One popular method, the Metropolis-Hastings scheme consists of the following procedure.

\begin{algorithm}[h]
\caption{Metropolis-Hastings algorithm}
Set $x_0 = 0, i=0$\\
Repeat ad nauseum:\\
\Indp
Increment $i$\\
Draw $x$ from a proposal distribution $q(x|x_{i-1})$\\
Set $\alpha(x_{}|x_{i-1}) = 1 - min(1, \frac{\pi(x)q(x_{i-1}|x)}{\pi(x_{i-1})q(x|x_{i-1})})$\\
Draw $u$ from a uniform density on $[0,1]$.
Set $x_i = x$ with probability $1 - \alpha$, i.e. if $u >\alpha$, and $x_i = x_{i-1}$ otherwise.\\
\end{algorithm}

Suppose this MCMC algorithm produces a chain $ x_1, x_2, x_3, ...$ of samples. Because the algorithm is stochastic, these samples can be viewed as realizations of random variables $X_1, X_2, X_3, ...$ with marginal density functions $f_1, f_2, f_3, $ etc. If you initialize deterministically, then $X_1$ is just a constant. Because $X_i$ is independent of past draws given $X_{i-1}$, we can write $f_i(x_i) = \int f_{i|i-1}(x_{i},x_{i-1})f_{i-1}(x_{i-1})dx_{i-1}$ using the conditional density of $X_i$ given $X_{i-1}$. Noting that $f_{i|i-1}$ doesn't depend on $i$, we can replace it with a function $K$ so that $f_i(x_i) = \int K(x_i, x_{i-1})f_{i-1}(x_{i-1})dx_{i-1}$. This function $K$, called the Markov kernel, is analogous to the transition probability matrices of discrete-space Markov chain theory. We refer to the linear operator of integrating against $K$ as $L$, so that $f_{i} = Lf_{i-1}$. The object of interest is the steady state of this operator, an eigenfunction $\pi$ that has eigenvalue $1$ so that for any $x$, $\pi(x) = \int K(x, t)\pi(t)dt$. In MCMC methods, chains are usually left to run until the Markov chain reaches its steady state. In BEMC, we approximate $L$, then compute $\pi$ from the approximation. 

\subsection{Stage one: approximating the kernel}
\label{sec:BEMC}
Our estimator is parametric, using a fixed set of functions $\{h_i\}_{i=1}^B$ from $\Omega$ to $\mathbb{R}$. We will choose them to be orthogonal with respect to an $L_2$ inner product, i.e. $\int_{\Omega} h_i(x)h_j(x)dx = 0$ when $i \neq j$. For $\Omega=\mathbb{R}^n$, we use Hermite functions, which are exponentially-weighted orthogonal polynomials. We will attempt to estimate a matrix $M$ in $\mathbb{R}^{B\times B}$ such that $L \approx \hat{L}$, where $(\hat{L}f)(x) =\sum_{i,j=1}^B h_i(x)M_{ij}\int h_j(x)f(x)dx$. Equivalently, we approximate $K$ as $\hat{K}(x,y) = \sum_{i,j=1}^B M_{ij} h_i(x)h_j(y)$. 

This approximation can imitate continuous kernels, i.e. situations where $\int K(x_i, x_{i-1})f_{i-1}(x_{i-1})dx_{i-1}$ can be done with respect to the Lebesgue measure. This presents an obstacle, because with positive probability, the Metropolis-Hastings algorithm will reject a proposed sample and stay in place. As a workaround, we approximate the kernel not of a single M-H iteration but of $\ell$ iterations for $\ell$ around 10 or 20. The probability of $\ell$ consecutive rejections is much smaller, forcing the true kernel closer to the subspace in which we approximate it. In section \ref{sec:BEMC-R}, we discuss a variant that explicitly models rejection events.

How will we choose $M$? Notice that the orthogonality of the basis functions implies $\int h_i(x)(\hat{L}h_j)(x)dx =  M_{ij}$. This can be written as an expectation $M_{ij} = E_{Lh_j}[h_i]$, which motivates us to sample from $Lh_j$ and approximate $M_{ij}$ as a sum. All we need to do is sample $z$ from $h_j$, run an M-H iteration on $z$ to get $w$, and retain $w$ as our sample. 

How do we ``sample'' from $h$, a basis function that sometimes takes negative values? How do we formally take an expectation? The important property to preserve is the law of large numbers: sample averages of should still converge to their expectation. We use a classic tactic from analysis. Let $h_+(x)$ be defined as $\frac{1}{c_+}\max(h(x), 0)$ and let $h_-(x)$ be defined as $\frac{-1}{c_-}\min(h(x), 0)$, with $c_+$ and $c_-$ chosen so $h_+$ and $h_-$ each integrate to one. Then define $E_{h}[f]$ as $c_+E_{h_+}[f]-c_-E_{h_-}[f]$. We can approximate this expectation by sampling $z_{n+}$ from $h_+, n=1...N_+$ and $z_{n-}, n=1...N_-$ from $h_-$. We would then compute $E_{h}[f] \approx \frac{c_+}{N_+}\sum f(z_{n+})-\frac{c_-}{N_-}\sum f(z_{n-})$. %The optimal allocation of samples between $h_+$ and $h_-$ minimizes the overall variance, $\frac{c_+^2}{N_+}Var_{h_+}[f] + \frac{c_-^2}{N_-}Var_{h_-}[f]$. To sample from $h_+$ and $h_-$, which may not have closed-form inverse CDF's, we employ rejection sampling. 

To take care of one last detail, suppose $\phi$ is $Lh$ for some $h$, and we can only sample from $\phi$ by running an M-H iteration on samples from $h$. We need to know we can sample from $\phi_+$ by sampling from $h_+$ and applying an M-H iteration. In fact, we can, because $L$ will not change the sign of a function. 

\begin{algorithm}[h]
\caption{BEMC algorithm--stage one}
Set $M$ to a matrix of all zeroes.\\
For $b_{in}  = 1:B$\\
\Indp
For $b_{out}  = 1:B$\\
\Indp
For $n = 1:N$\\
\Indp
Draw a sample $z_n$ from $h_{b_{in}}$.\\
Run the M-H sampler for $\ell$ rounds on $z_n$. Call the result $w_n$.\\
Increment $M_{b_{out}, b_{in}}$ by $h_{b_{in}(w_n)}/N$.\\
\Indm
\Indm
\Indm

\end{algorithm}

\subsection{BEMC-G, a Gibbs sampling variant}
\label{sec:BEMC-G}
This approximation can also be adapted to Gibbs sampling, a ubiquitous MCMC variant. 

\subsection{BEMC-R, a variant modeling rejections}
\label{sec:BEMC-R}

As we mention in section \ref{sec:BEMC}, our scheme is able to model continuous kernels. On the other hand, the Metropolis-Hastings algorithm sometimes rejects proposed samples, so its kernel will have a component shaped like a Dirac delta function. In this section, we introduce a variant of BEMC that explicitly models rejections by the sampler. 

Let us look at the Metropolis-Hastings kernel in more detail. Going back to the algorithm, the quantity $\alpha(x_{}|x_{i-1})$ is the probability of rejecting a move from $x_{i-1}$ to $x$. For convenience, let $\alpha(x_{i-1})$ denote the (overall) probability of rejecting a move from $x_{i-1}$. Splitting up the next draw as an alternative between moving and staying put, we can write $K(x_2, x_1) = \alpha(x_1)\delta_{x_1}(x_2) + (1-\alpha(x_1))r(x_2|x_1)$. In this expression, $r(x_2|x_1)$ is the conditional density of $x_2$ given that our move out of $x_1$ was not rejected. This is not the same as $q(x_2|x_1)$, since the lack of rejection informs us that we have more likely moved into a region of higher probability. To set up the last line below, define $D_{\alpha}$ from $\alpha$ so that $(D_{\alpha}f)(x)\equiv \alpha(x)f(x)$, and let $(L_{acc}f)(x)\equiv \int r(x|y)(1-\alpha(y))f(y)dy$. Then:
\begin{align*}
 f_2(x_2) &= \int K(x_2, x_1)f_1(x_1)dx_1 \\
&= \int (\alpha(x_1)\delta_{x_1}(x_2) + (1-\alpha(x_1))r(x_2|x_1))f_1(x_1)dx_1 \\
&= \int \alpha(x_1)\delta_{x_2}(x_1)f_1(x_1)dx_1 + \int (1-\alpha(x_1))r(x_2|x_1)f_1(x_1)dx_1 \\
&=  \alpha(x_2)f_1(x_2) + \int (1-\alpha(x_1))r(x_2|x_1)f_1(x_1)dx_1 \\
&=  (D_{\alpha}f_1)(x_2) + (L_{acc}f_1)(x_2) \\
\end{align*}

We can sample from a pdf proportional to $D_{\alpha}f$ by sampling $z$ from $f(x)$, then running an M-H iteration on $z$ to get $w$ and retaining the sample $z$ if $w \neq z$. We can sample from a pdf proportional to $L_{acc}f$ by doing nearly the same steps, but retaining $w$ if $w \neq z$. These facts will be useful as we attempt to estimate $L_{acc}$.

This time around, we will try to estimate a function $\hat{\alpha}$ and a matrix $M$ so that $\hat{\alpha}\approx{\alpha}$ and $L_{acc} \approx \hat{L}_{acc}$, where $(\hat{L}_{acc}f)(x) =\sum_{i,j=1}^B h_i(x)M_{ij}\int h_j(x)f(x)dx$. Even if the parameters were chosen optimally, $L$ may not take the same form as $\hat{L}_{\alpha}+\hat{L}_{M}$, so the estimate $\hat{\pi}$ will not be correct. \EMK{Need some results answering ``in what sense is your method correct?''}

For this variant, we need still need to estimate $M$ with the added complication of trying to infer $\hat{\alpha}$ at the same time. Fortunately, it is easy to tell when the sampler rejects and when it doesn't, and this provides a way to tease out information about $\alpha$. Suppose for a moment that we start the sampler at a point $z$ and it takes a single step to $w$. If $w \neq z$, then the sampler has shown less of a tendency to reject starting from $z$, and we label $z$ with a $0$. If $w = z$, we label $z$ with a $1$. Once the sample space is covered in zeroes and ones, there are many probabilistic classifier methods that could give an estimate of $\hat{\alpha}$, which at any given point is just the probability of labeling with a one. Meanwhile, whenever the sampler moves, we gain information about $L_{acc}$, and we can update $M$ as before. 

This strategy still throws away useful information. To see why, recall that the Metropolis-Hastings algorithm makes a proposal, computes an rejection probability, flips a proverbial coin with that probability, and then discards the rejection probability. When drawing a chain of samples, the rejection probability serves no further purpose, so discarding it is natural. In BEMC-R, though, it provides a more efficient estimate of $\alpha$. If the rejection probability when proposing a move to $w$ from $z$ is $p$, then the better procedure is to label $z$ with $p$. Likewise, instead of updating the estimate of $M_{ij}$ with sample of weight $1$ with probability $p$, we can update it with a sample of weight $p$.

\begin{algorithm}[h]
\caption{BEMC-R algorithm--stage one}
Set $M$ to $0$.\\ 
Set a scalar $W$ to zero. $W$ is the effective number of samples in an estimate of an entry of $M$.\\
Set $T = \{\}$. $T$ will be the training set for $\hat{\alpha}$.\\
For $b_{in}  = 1:B$\\
\Indp
For $b_{out}  = 1:B$\\
\Indp
For $n = 1:N$\\
\Indp
Draw a sample $z_n$ from $h_{b_{in}}$.\\
Draw a proposal $w_n$ and compute its rejection probability $p$.\\
Add $(z_n, p)$ to $T$.\\
Increment $M_{b_{out}, b_{in}}$ by $ph_{b_{in}(w_n)}$.\\
Increment $W$ by $p$.\\
\Indm
Divide $M_{b_{out}, b_{in}}$ by $W$.\\
\Indm
\Indm
Train $\hat{\alpha}$ on $T$.\\
\end{algorithm}



\subsection{Computing the steady state in BEMC-R}
 Given $\hat{M}$, $\hat{\alpha}$, and an initial state $f_0$, we want to compute $[D_{\hat{\alpha}}+\hat{L}_{acc}]^P(f_0)$ for some moderately high exponent $P$. To simplify the problem, suppose we set $f_0$ to $h_1$, one of the initial $B$ basis functions. Also, suppose that we restrict $\hat{\alpha}$ to a form where for any of our basis functions $h_i$, we can expand $\hat{\alpha}h_i$ as a sum $\sum_{i=1}^B c_i h_i$. Because of the orthogonality, computing $\hat{L}_{acc}(f_0)$ is simple: $\hat{L}_{acc}(f_0) = \sum_{i=1}^B \hat{L}_{acc}(c_i h_i) = \sum_{i=1}^B [\hat{M}c]_i h_i$. The difficulty lies in finding a representation of $D_{\hat{\alpha}}(f_0)$ in this basis, i.e. evaluating or quickly approximating integrals of the form $\int_{\Omega} h_i(x)h_j(x)\hat{\alpha}(x)dx$. \EMK{Maybe we'll choose a crafty form for $\hat{\alpha}$ and do this analytically. }

\section{Implementation details}
At this point, we will introduce a family of basis functions and begin to work in specifics. We will at first discuss the scenario where the sample space $\Omega$ is $\mathbf{R}$. %The Hermite polynomials have two commonly used forms; we we use the one called the probabilists' Hermite polynomials, usually denoted $\{He_n\}_{n=0}^{\infty}$. These functions are generated by setting $He_{n+1}=xHe_n-nHe_{n-1}$, with $He_0$ 
Our basis of choice, the Hermite functions, are generated by setting $\phi_{n+1}=\sqrt{\frac{2}{n+1}}\left[x\phi_n-\sqrt{\frac{n}{2}}\phi_{n-1}\right]$, with $\phi_0=\pi^{-\frac{1}{4}}e^{\frac{-x^2}{2}}$ and $\phi_1=\sqrt{2} \pi^{-\frac{1}{4}}xe^{\frac{-x^2}{2}}$. It has the property that $\int_{\mathbf{R}}\phi_i\phi_j=1$ if $i = j$ and $0$ otherwise. 
%abramowitz and stegun page 775

In order to fit the BEMC mold, we need to separate the positive and negative parts for each function, find their normalizing constants $c_+$ and $c_-$, and sample from densities proportional to the positive and negative parts. Luckily, the sign changes are few in number and easy to find: they're just the roots of the Hermite polynomials. \EMK{Only true if you make sure Hermite polynomials have no irreducible quadratic factors. Is there a closed form for the Hermite polynomial roots?} We need to integrate terms of the form $cx^k\exp(\frac{-x^2}{2})$, a task we can rewrite using gamma densities. Below, $Y$ is a Gamma random variable with shape $a = (k+1)/2$ and rate $b=-1/2$; $F_Y$ is its cumulative distribution function (CDF). The variables $x_0$ and $x_1$ are roots of whichever Hermite polynomial we are working with.

\begin{align*}
\int_{x_0}^{x_1} cx^k\exp(\frac{-x^2}{2})dx 
& =  \int_{x_0^2}^{x_1^2} \frac{c}{2}y^{(\frac{k-1}{2})}\exp(\frac{-y}{2})dy \\
& =  \frac{c}{2} \frac{\Gamma(a)}{b^a} \int_{x_0^2}^{x_1^2} \frac{b^a}{\Gamma(a)}y^{(a-1)}\exp(-by)dy \\
& =  \frac{c}{2} \frac{\Gamma(a)}{b^a} (F_Y(x_1^2)-F_Y(x_1^2))\\
\end{align*}

Another way to phrase this trick: the square root of a Gamma random variable with the right parameters has a density function proportional to $x^k\exp(\frac{-x^2}{2})$. This is also called the Nakagami distribution, and the \texttt{R} package \texttt{VGAM} has functions \texttt{rnaka()} (for sampling) and \texttt{pnaka()} for evaluating the CDF. To match $x^k\exp(\frac{-x^2}{2})$, the call should be \texttt{pnaka(x, shape=(k+1)/2, rate=k+1)}.

To sample from $\phi_{j+}$, the positive part of the $j$th Hermite function, we can generate uniform random numbers and then apply $F^{-1}_{j+}$, where $F_{j+}(x) = \int_{-\infty}^x \phi_{j+}(t) dt$. Given a list of Hermite polynomial roots $x_0, ... x_j$ so that $\phi_{j+}$ switches from $0$ to $\phi_{j}/c_+$ or vice versa, $F_{j+}(x) = \int_{-\infty}^{x_0} \phi_{j+}(t) dt + 0 + \int_{x_1}^{x_2} \phi_{j+}(t)dt + ... + \int_{x_j}^{x} \phi_{j+}(t) dt $. Actually, the zero terms depend on the degree: if $j$ is odd, then $\phi_{j}$ is positive at negative infinity, and if $j$ is odd, $\phi_{j}$ is negative at negative infinity. Since we know the roots beforehand, everything but the last term can be computed offline. This suggests an algorithm for quickly finding $F^{-1}_{j+}(x)$.  

\begin{algorithm}[h]
\caption{Sampling from $\phi_{J±}$. Suppose the roots of $\phi_{J}$ are $x_0, ... x_J$ and the coefficient of $x^k\exp(\frac{-x^2}{2})$ in $\phi_{J}$ is $v_k$. $G$ is the CDF of the Nakagami distribution.}

For $j$ from $0$ to $J+1$:\\
\Indp
For $k$ from $0$ to $J$:\\
\Indp
Set $I_{jk}$ to {$G(x_{j-1}, shape=(k+1)/2, rate=k+1)$} - {$G(x_j, shape=(k+1)/2, rate=k+1)$}. In place of $G(x_{-1}, shape=(k+1)/2$, use $0$, and use $1$ instead of $G(x_{J+1}, shape=(k+1)/2$.\\
\Indm
\Indm
Draw a random number $u \sim unif(0,1)$.

\Indp
\Indm
\end{algorithm}

\EMK{Also considered using rejection sampling from Nakagami distribution. This might scale better with the dimension of the parameter space.} 

 

\section{Proof of Correctness}
In stage $1$, we will suffer some error while estimating $M$ and $\alpha$ from. The true transition kernel cannot always be represented in the finite-dimensional form we impose, which is a second source of error. In stage two of BEMC-R, we incur a third source, namely that $D_{\hat{\alpha}}(f_0)$ does not necessarily take the form $\sum_{i=1}^B c_i h_i$, especially if $\hat{\alpha}$ is also represented in the same basis. For example, if the basis were Gaussian-weighted polynomials up to degree 15, the degree of $D_{\hat{\alpha}}(f_0)$ would quickly exceed 15, and the variance of the Gaussian would change. 

%==== Bib files and style =======
\bibliographystyle{splncs}
\bibliography{eric_kernfeld_biblio}

\end{document}
